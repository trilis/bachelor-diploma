%!TEX TS-program = xelatex
\documentclass{beamer}

\usepackage{HSE-theme/beamerthemeHSE-en} % Load HSE theme

%%% Fonts 
\usepackage{fontspec}
\defaultfontfeatures{Ligatures={TeX},Renderer=Basic}
\setmainfont[Ligatures={TeX,Historic}]{Arial} % install Myriad Pro or replace with Arial
\setsansfont{Arial}  % install Myriad Pro or replace with Arial
\setmonofont{Courier New}

\usepackage{multicol}       % Multiple columns
\graphicspath{{images/}}    % Images folder

\usepackage[
backend=biber,
style=apa,
citestyle=authoryear
]{biblatex}
\addbibresource{presentation.bib}

\usepackage{amssymb}% http://ctan.org/pkg/amssymb
\usepackage{pifont}% http://ctan.org/pkg/pifont
\newcommand{\cmark}{\textcolor{green}{\ding{51}} \hspace{4pt}}%
\newcommand{\xmark}{\textcolor{red}{\ding{55}} \hspace{4pt}}%



%%% Author and speech
\title[Retrofitting Implicit Modules for 1ML]{Retrofitting Implicit Modules for 1ML} 
\subtitle{Research Proposal}
\author[Alexey Trilis]{Alexey Trilis, group BPM171 \\
Scientific Advisor: Daniil Berezun, PhD}
%, Research Associate at ITMO University
\institute[Higher School of Economics]{National Research University Higher School of Economics \\
St. Petersburg School of Physics, Mathematics, and Computer Science}
%\\ Department of Applied Mathematics and Business Informatics
\date{Saint Petersburg, 2021}

\begin{document}    % Document begins
\frame[plain]{\titlepage}   % Title frame

\section{Just some text}
\subsection{Subtitle}

\begin{frame}{Outline}
\begin{itemize}
    \item Background Information
    \item Motivation
    \item Research Objective
    \item Literature Review
    \item Methodology
    \item Anticipated Results
\end{itemize}
\end{frame}

\begin{frame}{Background Information 1/3}{Ad hoc polymorphism}
\textbf{Ad hoc polymorphism}
\begin{itemize}
\item Functions with the same name having different semantics depending on argument types.
\item Such as \texttt{+} or \texttt{print}
\end{itemize}

\textbf{Approaches}
\begin{itemize}
\item \textit{Type classes}, like in Haskell (\cite{adhoc})
\item \textit{Implicits}, like in Scala (\cite{implicits})
\end{itemize}

\end{frame}

\begin{frame}{Background Information 2/3}{ML languages}
\textbf{ML language family}
\begin{itemize}
    \item ML, SML, OCaml, F\#, many others
    \item Powerful type inference
    \item Advanced module system
    \item Applications: compiler development, static program analysis, automatic theorem proving, financial systems, web development
\end{itemize}
\textbf{No ad hoc polymorphism!}
\begin{itemize}
    \item Need to use \texttt{+}, \texttt{+.}, \texttt{print\_int}, \texttt{print\_string}, etc.
    \item Undesirable verbosity
\end{itemize}
\end{frame}

\begin{frame}{Background Information 3/3}{1ML}
\textbf{ML modules}
\begin{itemize}
    \item Two languages: core and module
    \item Module language is more powerful, built on dependent-type machinery (\cite{dependent_types})
    \item Module language is verbose and difficult to integrate with core language
\end{itemize}
\textbf{1ML}
\begin{itemize}
    \item Later research showed that dependent types are not actually needed (\cite{fing})
    \item 1ML (\cite{1ml}) is an experimental ML dialect in which core and module languages are united
\end{itemize}
\end{frame}

\begin{frame}{Motivation}
\begin{itemize}
    \item Extending ML with implicits is highly desirable by users
    \item However, it is not an easy task
    \item We limit ourselves to a simpler language
    \item However, because of some 1ML's design choices, we can achieve new results not currently possible in more mature ML languages
    \item Also this project can be viewed as test of expressiveness for 1ML
\end{itemize}
\end{frame}

\begin{frame}{Research Objective}
\textbf{Research objective.} 
Design implicit modules for 1ML.
\textbf{Subgoals}
\begin{itemize}
    \item Reproduce all results already described for OCaml
    \item Improve completeness of these results by paying special attention to resolve order and unification
    \item Compare implementation with OCaml one
\end{itemize}
\end{frame}

\begin{frame}{Literature Review}
\begin{itemize}
    \item \cite{ml_typeclasses}~---~designing type classes for ML
    \begin{itemize}
        \item Type classes require \textit{canonicity}
        \item In ML languages, canonicity is impossible to achieve
        \item Without canonicity, authors impose rather severe restrictions
    \end{itemize}
    \item \cite{white}~---~designing implicits for OCaml
    \begin{itemize}
        \item Only prototype implementation, merging in OCaml is years away
        \item Both practical and theoretical difficulties
        \item Needs \textit{higher-order unification}, which OCaml currently lacks
    \end{itemize}
\end{itemize}
\end{frame}

\begin{frame}{Methodology}
\begin{itemize}
    \item Resolving of implicits must be done simultaneously with type inference
    \item Delaying resolve while it is possible
    \item Recursive search based on a set of type constraints
    \item Termination check
    \item Retrying ambiguous resolve with new information
\end{itemize}
\end{frame}

\begin{frame}{Anticipated Results}
\begin{itemize}
    \item Implementated implicits for 1ML, as part of 1ML prototype compiler
    \item We test our implementation by comparing its strength against OCaml solution
    \item Our solution works in all cases where OCaml solution works
    \item There are cases (mostly related to order and retrying) where our solution works while OCaml solution does not
    \item We support \textit{implicit functors}, which are listed By OCaml solution's authors as a natural future work  
    \item \textcolor{gray}{Results related to unification}
\end{itemize}
\end{frame}

\begin{frame}[t, allowframebreaks]
\frametitle{References}
\printbibliography
\end{frame}


\begin{frame}[c]
\begin{center}
\frametitle{\LARGE Thank you for your attention!}   

{\LARGE \inserttitle}

{\insertsubtitle}

\bigskip

{\insertauthor} 

\bigskip\bigskip

{\insertinstitute}

\bigskip\bigskip

{\large \insertdate}
\end{center}
\end{frame}

\end{document}