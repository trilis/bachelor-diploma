%% Простая презентация с примером включения программного кода и
%% пошаговых спецэффектов
\documentclass{beamer}
\usetheme{SPbAU}
%\useoutertheme{infolines}
\usepackage{fontspec}
\usepackage{xunicode}
\usepackage{xltxtra}
\usepackage{xecyr}
\usepackage{hyperref}
\setmainfont[Mapping=tex-text]{DejaVu Serif}
\setsansfont[Mapping=tex-text]{DejaVu Sans}
\setmonofont[Mapping=tex-text]{DejaVu Sans Mono}
\usepackage{polyglossia}
\usepackage{csquotes}
\setdefaultlanguage{russian}
\usepackage{graphicx}
\usepackage{listings}
\usepackage{xpatch}
\lstdefinestyle{highlight}{
  basicstyle=\footnotesize\ttfamily\color{black},
  keywordstyle=\bfseries\color{black},
  commentstyle=\itshape\color{black},
}
\lstdefinestyle{base}{
  belowcaptionskip=1\baselineskip,
  breaklines=true,
  xleftmargin=\parindent,
  showstringspaces=false,
  basicstyle=\footnotesize\ttfamily\color{black!40},
  keywordstyle=\bfseries\color{black!40},
  commentstyle=\ttfamily\itshape\color{black!40},
  stringstyle=\color{red},
  numbers=left,
  numbersep=5pt,
  numberstyle=\tiny\color{gray},
  morecomment=[s]{(*}{*)},
  texcl=true,
  literate={п}{\cyrp}1
             {и}{\cyri}1,
  moredelim=**[is][\only<+>{\color{black}\lstset{style=highlight}}]{§}{§},
  morekeywords={implicit, module, struct, sig, val}
}
\lstdefinestyle{base2}{
  belowcaptionskip=1\baselineskip,
  breaklines=true,
  xleftmargin=\parindent,
  showstringspaces=false,
  basicstyle=\footnotesize\ttfamily\color{black},
  keywordstyle=\bfseries\color{black},
  commentstyle=\ttfamily\itshape\color{black},
  stringstyle=\color{red},
  numbers=left,
  numbersep=5pt,
  numberstyle=\tiny\color{gray},
  morecomment=[l]{(*},
  texcl=true,
  literate={п}{\cyrp}1
             {и}{\cyri}1,
  moredelim=**[is][\only<+>{\color{black}\lstset{style=highlight}}]{§}{§},
  morekeywords={implicit, module, struct, sig, val}
}
\newcommand{\backupbegin}{
   \newcounter{framenumberappendix}
   \setcounter{framenumberappendix}{\value{framenumber}}
}
\newcommand{\backupend}{
   \addtocounter{framenumberappendix}{-\value{framenumber}}
   \addtocounter{framenumber}{\value{framenumberappendix}} 
}

\usepackage[backend=biber, style=authortitle, doi=false, url=false, isbn=false]{biblatex}
\xapptobibmacro{cite}{\setunit{\nametitledelim}\printfield{year}}{}{}
\addbibresource{presentation-short.bib}

\begin{document}
\title[Неявные модули в 1ML]{Переоснащение неявных модулей для языка 1ML }
\author[Трилис А.А.]{Трилис Алексей Андреевич\\{\footnotesize\textcolor{gray}{научный руководитель: к.ф.-м.н. Д.А. Березун}}}
\institute{НИУ ВШЭ --- Санкт-Петербург}
\date{11 июня 2021 г.}
\frame{\titlepage}

\begin{frame}\frametitle{Введение}
\begin{itemize}
  \item Ad-hoc-полиморфизм~--- свойство языка, позволяющее функциям иметь различную семантику в зависимости от типов аргументов
  \item Часто достигается перегрузкой, но в языках с мощным выводом типов нужны более сложные методы
  \item В семействе языков ML ad-hoc-полиморфизм до сих пор не поддерживается
  \item \texttt{+} для \texttt{int}, \texttt{+.} для \texttt{float}

  \texttt{print\_int}, \texttt{print\_float}, \texttt{print\_string}, ...
\end{itemize}
\end{frame}

\lstset{language=haskell}
\begin{frame}[fragile]\frametitle{Обзор литературы}
\begin{itemize}
  \item Согласованность~--- каждая типизация программы должна приводить к одной и той же семантике
  \item Каноничность~--- в области видимости не более одного экземпляра для каждого типа 
  \item Решения с каноничностью
  \begin{itemize}
    \item Haskell, Rust
    \item В ML невозможно проверить каноничность
    \item Есть решение для ML \footcite{ml_typeclasses}, но оно вводит серьёзные ограничения
  \end{itemize}
  \item Решения без каноничности
  \begin{itemize}
    \item Scala
    \item Решение для OCaml \footcite{white} в основной язык интегрировать пока не получилось
    \item Система проверки типов недостаточно сильна
  \end{itemize}
\end{itemize}
\end{frame}

\lstset{language=caml}
\begin{frame}[fragile]\frametitle{Неявные модули}
\begin{lstlisting}[style=base]
§module type Show = sig
  type t
  val show : t -> string
end
§§
implicit module Show_int = struct 
  type t = int
  let show x = string_of_int x
end
§§
implicit module Show_list {S : Show} = struct 
  type t = S.t list
  let show x = string_of_list S.show x
end
§§
let show {S : Show} x = S.show x 

§§show 5 (* show {Show_int} 5 *)
show [1;2;3] (* show {Show_list {Show_int}} [1;2;3] *)§
\end{lstlisting}
\end{frame}

\begin{frame}\frametitle{1ML}
\begin{itemize}
  \item В ML исторически язык разделён на основной язык и более мощный и избыточный модульный язык
  \item Интеграция этих слоёв затруднена
  \item Экспериментальный диалект 1ML\footcite{1ml} решает эту проблему: в нём нет существенного различия между основным языком и модулями
  \item Предположительно, такой подход позволит добавить новую функциональность, например, ad-hoc-полиморфизм
\end{itemize}
\end{frame}

\begin{frame}\frametitle{Цель и задачи}
\textbf{Цель}: Разработать поддержку неявных модулей для языка 1ML, которое будет более полным, чем аналогичное решение для OCaml, в том числе включать поддержку неявных аргументов для функторов

\textbf{Задачи}:
\begin{itemize}
  \item Реализация неявных модулей для 1ML, повторяющих функциональность решения для OCaml
  \item Разработка алгоритма, позволяющего полно и эффективно осуществлять вставку неявных модулей
  \item Поддержка неявных аргументов для функторов
  \item Сравнение с решением для OCaml
\end{itemize}
\end{frame}

\begin{frame}\frametitle{Общая схема}
\begin{itemize}
  \item На месте, где должен стоять модуль, подставим неявную переменную
  \item Пока не знаем значение, но знаем тип
  \item Отложим определение этого модуля, будем обрабатывать несколько неявных переменных за раз
  \item Представим текущее состояние поиска как набор ограничений на тип
  \item Переберём все доступные модули, если подходит под ограничения~--- подставим
  \item Если это функтор, то запустимся рекурсивно с новыми ограничениями
\end{itemize}
\end{frame}

\lstset{language=caml}
\begin{frame}[fragile]\frametitle{Детали решения}
\begin{itemize}
  \item Проверка завершаемости
  \begin{itemize}
    \item В рекурсии проверяем, что хотя бы одно ограничение уменьшилось, а остальные не стали больше
  \end{itemize}
  \item В какой момент разрешаем накопившиеся переменные?
  \begin{itemize}
    \item Дойдя до объявления верхнего уровня
    \item Но в теории можно в любой момент
  \end{itemize}
  \item Локальные неявные модули
  \begin{itemize}
    \item В момент разрешения модули могут выйти из контекста
    \item Храним дерево из неявных модулей и побочной информации
  \end{itemize}
\end{itemize}
\end{frame}

\begin{frame}\frametitle{Порядок разрешения. Мотивация}
\begin{itemize}
  \item В решении для OCaml неявные переменные разрешаются в некотором фиксированном порядке
  \item Это уменьшает полноту решения
  \item В этой работе найдено, что решение для OCaml не работает в простых случаях: невозможно реализовать сложение переменных типа \texttt{int} и \texttt{float}
  \item Будем запускать разрешение неявных переменных несколько раз, с появлением новой информации
\end{itemize}
\end{frame}

\begin{frame}\frametitle{Порядок разрешения. Алгоритм}
\begin{itemize}
  \item Обрабатываем неявные переменные в следующем порядке:
  \begin{enumerate}
    \item Независимые от других ещё не разрешённых
    \item Не обработанные раннее
    \item Те, с последней обработки которых была получена новая информация
  \end{enumerate}
  \item Несколько эвристик
  \item В случаях, которые были поддержаны раньше, работает столь же эффективно
\end{itemize}
\end{frame}

\lstset{language=caml}
\begin{frame}[fragile]\frametitle{Неявные аргументы для функторов}
Есть \texttt{Show\_list1} и \texttt{Show\_list2}, нужно выбрать из них явно
\begin{lstlisting}[style=base2]
show {Show_list1 {Show_pair {Show_int} {Show_bool}}}
  [(1, true); (2, false)]

(* Слишком длинно. В OCaml можно только так *)

show {Show_list1 [_]} [(1, true); (2, false)]

(* В 1ML можно поддержать такое *)
\end{lstlisting}
Это достигнуто за счёт того, что в 1ML различие между функциями и функторами существенно меньше
\end{frame}

\begin{frame}\frametitle{Сравнение}
\begin{itemize}
  \item Тест~--- две аналогичные программы на OCaml и на 1ML
  \item Проверяем, корректна ли проверка типов для теста в каждом из языков
  \item Сейчас около 30 тестов
\end{itemize}
\end{frame}

\begin{frame}\frametitle{Результаты}
\begin{itemize}
    \item Реализованы неявные модули как расширение компилятора языка 1ML
    \item Решение работает на тестах, на которых работает решение для OCaml
    \item Также работает на тестах, которые в OCaml не поддерживаются
    \begin{itemize}
        \item \textbf{Порядок разрешения}

        Идея алгоритма может быть использована в OCaml
        \item \textbf{Неявные аргументы для функторов}

        Результат достигнут из-за особенностей 1ML
    \end{itemize}
    \item Работа представляет дополнительный аргумент в пользу диалектов с однородным подходом к модулям
\end{itemize}
Репозиторий: \url{github.com/trilis/1ml}
\end{frame}

\appendix
\backupbegin

\lstset{language=caml}
\begin{frame}[fragile]\frametitle{Детали алгоритма. Примеры}
Пример модуля, в присутствии которого поиск не будет завершаться:
\begin{lstlisting}[style=base2]
implicit module Show_it {S : Show} = struct
  type t = S.t
  let show = S.show
end
\end{lstlisting}
Пример локального неявного модуля:
\begin{lstlisting}[style=base2]
let f = show 5 ^ " " ^
  (let implicit module Show_float = struct
      type t = float
      let show x = string_of_float x
   end in show 3.14)
\end{lstlisting}
\end{frame}

\lstset{language=caml}
\begin{frame}[fragile]\frametitle{Порядок разрешения. Пример}
\begin{lstlisting}[style=base2]
module type Num = sig
  type t and u and res
  val ( + ) : t -> u -> res
end;;

let ( + ) {N : Num} = N.( + );;

implicit module Float_Float = struct
  type t = float and u = float and res = float
  let ( + ) = ( +. )
end;;
implicit module Int_Float = struct
  type t = int and u = float and res = float
  let (+) l r = (float_of_int l) +. r
end;;

(* Int\_Int и Float\_Int пропущены для краткости *)

print_float (1 + 1.1 + 2.5);; (* неоднозначность! *)
\end{lstlisting}
\end{frame}

\lstset{language=caml}
\begin{frame}[fragile]\frametitle{Порядок разрешения. Эвристики}
\begin{enumerate}
    \item Независимые от других ещё не разрешённых
    \item Не обработанные раннее
    \item Те, с последней обработки которых была получена новая информация
  \end{enumerate}
\begin{itemize}
  \item В (2) обработать сначала неявные переменные с меньшим числом типовых переменных
  \item В (3) обработать сначала те, про которых стало известно больше новой информации
\end{itemize}
\end{frame}

\lstset{language=caml}
\begin{frame}[fragile]\frametitle{Невозможность каноничности в OCaml}
\begin{lstlisting}[style=base2]
module F (X : Show) = struct
  implicit module S = X
end

implicit module Show_int = struct
  type t = int
  let show = string_of_int
end

module M = struct
  type t = int
  let show _ = "An int"
end

module N = F(M)
\end{lstlisting}
\end{frame}


\begin{frame}\frametitle{Ограничения на модульные классы типов}
\begin{itemize}
  \item Неявные модули могут быть объявлены только на верхнем уровне
  \item Все модули на верхнем уровне должны быть явно типизированы
  \item На верхнем уровне могут находиться только модули
  \item Все неявные модули должны определять тип \texttt{t}, по которому будет проходить унификация 
\end{itemize}
\end{frame}


\backupend
\end{document}
