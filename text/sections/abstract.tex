\documentclass[../diploma.tex]{subfiles}
\begin{document}

Отсутствие ad-hoc-полиморфизма в языках семейства ML может считаться существенным недостатком. Цель данной работы --- расширить экспериментальный язык 1ML ad-hoc-полиморфизмом с помощью неявных модулей. Язык 1ML отличается минимальным и однородным подходом к системе модулей ML и отсутствием зависимых типов, что значительно упрощает некоторые задачи, возникающие при разработке ad-hoc-полиморфизма. Предлагаемое решение основывается на прототипе Вайта и других для языка OCaml, при этом является более полным. В решении поддержаны неявные аргументы для функторов, которые до этого не поддерживались ни в одной работе о ML, а также разработан алгоритм, эффективно находящий корректный порядок вывода неявных модулей.

\vspace*{\fill}

Ключевые слова: неявные параметры, неявные модули, имплициты, классы типов, язык модулей ML, OCaml, 1ML. 

\newpage

The lack of ad hoc polymorphism in languages ​​of the ML family can be considered a significant problem. The goal of this work is to extend the experimental language 1ML with ad hoc polymorphism using implicit modules. 1ML is distinguished by a minimal and uniform approach to the ML module system and the absence of dependent types. These features greatly simplify some of the problems that arise when developing ad hoc polymorphism. Our solution is based on the prototype of White et al. for the OCaml language and is more complete compared to it. This work supports implicit arguments for functors, which were not supported in previous literature on ML. Also, we present an algorithm that efficiently finds the correct order to resolve implicit modules.
\vspace*{\fill}

Keywords: implicit parameters, implicit modules, type classes, ML modules, OCaml, 1ML.

\end{document}
