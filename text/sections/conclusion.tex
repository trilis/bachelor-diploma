
\documentclass[../diploma.tex]{subfiles}
\begin{document}

В рамках данной работы были достигнуты следующие результаты: 

\begin{itemize}
	\item Разработано решение, добавляющее в язык 1ML ad-hoc-полиморфизм с помощью неявных модулей. Это решение было реализовано на языке OCaml в качестве расширения компилятора 1ML. Получившийся расширенный компилятор можно найти по адресу \href{https://github.com/trilis/1ml}{github.com/trilis/1ml};
	\item Поддержаны неявные аргументы для функторов. Эта функциональность не была поддержана в прототипе для OCaml, но авторы решения для OCaml упомянули добавление данной функциональности как интересное направление для дальнейших исследований. Этот результат достигнут за счёт использования языка 1ML, а именно за счёт уникального подхода к системе модулей ML в этом языке, и может восприниматься как дополнительный аргумент в пользу диалектов с однородным подходом к модулям;
	\item Показано с приведением явных примеров, что решение для OCaml может не работать на достаточно простых программах. Исследована причина, по которым это происходит, а именно неоптимальный порядок вывода модулей. Предложено и реализовано решение, которое является более полным по сравнению с решением для OCaml и при этом почти не проигрывает ему в эффективности. Это решение может с некоторой доработкой быть применено и в прототипе для OCaml;
	\item Осуществлена апробация решения с помощью запуска модифицированного компилятора на программах, содержащих неявные модули. Проверялось, что программы корректно типизируются. В основу тестового набора легли примеры из статей по теме и тестовый набор прототипа для OCaml. Апробация показала, что решение работает на тестах, на которых работает решение для OCaml. Также предлагаемое решение работает на тестах, которые в OCaml не поддерживаются, связанных с порядком разрешения и неявными аргументами для функторов, то есть является более полным.
\end{itemize}

\end{document}