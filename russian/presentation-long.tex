%% Простая презентация с примером включения программного кода и
%% пошаговых спецэффектов
\documentclass{beamer}
\usetheme{SPbAU}
%\useoutertheme{infolines}
\usepackage{fontspec}
\usepackage{xunicode}
\usepackage{xltxtra}
\usepackage{xecyr}
\usepackage{hyperref}
\setmainfont[Mapping=tex-text]{DejaVu Serif}
\setsansfont[Mapping=tex-text]{DejaVu Sans}
\setmonofont[Mapping=tex-text]{DejaVu Sans Mono}
\usepackage{polyglossia}
\usepackage{appendixnumberbeamer}
\setdefaultlanguage{russian}
\usepackage{graphicx}
\usepackage{listings}
\usepackage{xpatch}
\lstdefinestyle{mycode}{
  belowcaptionskip=1\baselineskip,
  breaklines=true,
  xleftmargin=\parindent,
  showstringspaces=false,
  basicstyle=\footnotesize\ttfamily,
  keywordstyle=\bfseries,
  commentstyle=\itshape\color{gray!40!black},
  stringstyle=\color{red},
  numbers=left,
  numbersep=5pt,
  numberstyle=\tiny\color{gray},
}
\lstset{escapechar=@,style=mycode}


\usepackage[backend=biber, style=authortitle, doi=false, url=false]{biblatex}
\xapptobibmacro{cite}{\setunit{\nametitledelim}\printfield{year}}{}{}
\addbibresource{presentation-long.bib}

\begin{document}
\title[Неявные модули в 1ML]{Переоснащение неявных модулей для языка 1ML }
\author[Трилис А.А.]{Трилис Алексей Андреевич\\{\footnotesize\textcolor{gray}{научный руководитель: к.ф.-м.н. Д.А. Березун}}}
\institute{НИУ ВШЭ --- Санкт-Петербург}
\frame{\titlepage}

\begin{frame}\frametitle{Введение. Ad hoc полиморфизм}
\begin{itemize}
  \item Ad hoc полиморфизм~--- свойство языка, позволяющее функциям иметь различную семантику в зависимости от типов аргументов
  \item Например, \texttt{print} и \texttt{+}
  \item В процедурных и объектно-ориентированных языках обычно достигается перегрузкой
  \item Но в языках с мощным выводом типов нужны более сложные методы
\end{itemize}
\end{frame}

\begin{frame}\frametitle{Введение. Семейство языков ML}
\begin{itemize}
  \item ML, OCaml, SML, F\#, ...
  \item Функциональные языки с мощным выводом типов
  \item Активно используется в разработке и исследовании языков программирования
  \item А также в верификации, финансах, веб-разработке и других областях
  \item Отсутствует ad hoc полиформизм:
  \begin{itemize}
    \item \texttt{+} для \texttt{int}, \texttt{+.} для \texttt{float}
    \item \texttt{print\_int}, \texttt{print\_float}, \texttt{print\_string}, ...
  \end{itemize}
\end{itemize}
\end{frame}

\begin{frame}\frametitle{Введение. Модули в ML}
\begin{itemize}
  \item Продвинутая система модулей, основанная на теории зависимых типов \footcite{dependent_types}
  \item Язык разделён на два слоя: основной и модульный
  \item Модульный язык более мощный, но требует 
\end{itemize}
\end{frame}

\begin{frame}\frametitle{Введение. 1ML}
\begin{itemize}
  \item Было показано \footcite{fing}, что модули могут быть выражены и без использования теории зависимых типов
  \item В результате этих исследований был создан экспериментальный диалект 1ML \footcite{1ml}
\end{itemize}
\end{frame}

\lstset{language=haskell}
\begin{frame}[fragile]\frametitle{Обзор литературы. Классы типов}
\begin{lstlisting}
class Show a where
  show :: a -> String

instance Show Int where
  show = showSignedInt

show_twice x = show x ++ show x

show_twice : Show a => a -> string
\end{lstlisting}
\begin{itemize}
  \item Впервые в Haskell\footcite{adhoc}, затем в Agda, Rust, ...
  \item Требуется каноничность~--- не более одного экземпляра для каждого типа
\end{itemize}
\end{frame}

\lstset{language=scala}
\begin{frame}[fragile]\frametitle{Обзор литературы. Неявные параметры}
\begin{lstlisting}
trait Showable [T] { def show (x: T): String }

implicit object IntShowable extends Showable [Int] {
  def show (x: Int) = x.toString
}

def show [T](x : T)(implicit s: Showable [T]): String = {
  s.show(x)
}

show(7)(IntShowable)
show(7)
\end{lstlisting}
Впервые в Scala\footcite{implicits}
\end{frame}

\begin{frame}\frametitle{Обзор литературы. Модулярные классы типов}
\begin{itemize}
  \item 
\end{itemize}
\end{frame}

\begin{frame}\frametitle{Обзор литературы. Неявные модули}
\begin{itemize}
  \item 
\end{itemize}
\end{frame}

\lstset{language=caml}
\begin{frame}[fragile]\frametitle{Обзор литературы. Неявные модули}
\begin{lstlisting}
module type Show = sig
  type t
  val show : t -> string
end

implicit module Show_int = struct
  type t = int
  let show x = string_of_int x
end

implicit module Show_list {S : Show} = struct
  type t = S.t list
  let show x = string_of_list S.show x
end

let show {S : Show} x = S.show x

show 5 (* show {Show_int} 5 *)
show [1;2;3] (* show {Show_list (Show_int)} [1;2;3] *)
\end{lstlisting}
\end{frame}

\begin{frame}\frametitle{Цель и задачи}
Цель:
\begin{itemize}
  \item Дополнить язык 1ML поддержкой неявных модулей
\end{itemize}
Задачи:
\begin{itemize}
  \item Реализация неявных модулей, повторяющих функциональность решения для OCaml
  \item Расширение решения новой функциональностью
  \item Тестирование и сравнение с решением для OCaml
\end{itemize}
\end{frame}

\begin{frame}\frametitle{Общая схема}

\end{frame}

\begin{frame}\frametitle{Поиск модулей}

\end{frame}

\begin{frame}\frametitle{Генерализация типов}

\end{frame}

\begin{frame}\frametitle{Проверка завершаемости}

\end{frame}

\begin{frame}\frametitle{Локальные неявные модули}

\end{frame}

\begin{frame}\frametitle{Порядок разрешения неявных модулей}

\end{frame}

\begin{frame}\frametitle{Неявные функторы}

\end{frame}

\begin{frame}\frametitle{Тестирование и сравнение}

\end{frame}

\begin{frame}\frametitle{Результаты}
\begin{itemize}
    \item
\end{itemize}
\end{frame}

\appendix

\begin{frame}[allowframebreaks]\frametitle{Ссылки}
\printbibliography
\end{frame}

\end{document}
