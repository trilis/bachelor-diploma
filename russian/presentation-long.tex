%% Простая презентация с примером включения программного кода и
%% пошаговых спецэффектов
\documentclass{beamer}
\usetheme{SPbAU}
%\useoutertheme{infolines}
\usepackage{fontspec}
\usepackage{xunicode}
\usepackage{xltxtra}
\usepackage{xecyr}
\usepackage{hyperref}
\setmainfont[Mapping=tex-text]{DejaVu Serif}
\setsansfont[Mapping=tex-text]{DejaVu Sans}
\setmonofont[Mapping=tex-text]{DejaVu Sans Mono}
\usepackage{polyglossia}
\usepackage{appendixnumberbeamer}
\setdefaultlanguage{russian}
\usepackage{graphicx}
\usepackage{listings}
\usepackage{xpatch}
\newcommand{\commentfont}{\ttfamily}
\lstdefinestyle{mycode}{
  belowcaptionskip=1\baselineskip,
  breaklines=true,
  xleftmargin=\parindent,
  showstringspaces=false,
  basicstyle=\footnotesize\ttfamily,
  keywordstyle=\bfseries,
  commentstyle=\itshape\color{gray!40!black},
  stringstyle=\color{red},
  numbers=left,
  numbersep=5pt,
  numberstyle=\tiny\color{gray},
  escapeinside=§§,
  escapebegin=\begin{russian}\commentfont,
  escapeend=\end{russian},
}
\lstset{escapechar=@,style=mycode}


\usepackage[backend=biber, style=authortitle, doi=false, url=false]{biblatex}
\xapptobibmacro{cite}{\setunit{\nametitledelim}\printfield{year}}{}{}
\addbibresource{presentation-long.bib}

\begin{document}
\title[Неявные модули в 1ML]{Переоснащение неявных модулей для языка 1ML }
\author[Трилис А.А.]{Трилис Алексей Андреевич\\{\footnotesize\textcolor{gray}{научный руководитель: к.ф.-м.н. Д.А. Березун}}}
\institute{НИУ ВШЭ --- Санкт-Петербург}
\frame{\titlepage}

\begin{frame}\frametitle{Введение. Ad hoc полиморфизм}
\begin{itemize}
  \item Ad hoc полиморфизм~--- свойство языка, позволяющее функциям иметь различную семантику в зависимости от типов аргументов
  \item Например, \texttt{print} и \texttt{+}
  \item В процедурных и объектно-ориентированных языках обычно достигается перегрузкой
  \item Но в языках с мощным выводом типов нужны более сложные методы
\end{itemize}
\end{frame}

\begin{frame}\frametitle{Введение. Семейство языков ML}
\begin{itemize}
  \item ML, OCaml, SML, F\#, ...
  \item Функциональные языки с мощным выводом типов
  \item Продвинутая система модулей, основанная на теории зависимых типов \footcite{dependent_types}
  \item Активно используется в разработке и исследовании языков программирования
  \item А также в верификации, финансах, веб-разработке и других областях
  \item Отсутствует ad hoc полиформизм:
  \begin{itemize}
    \item \texttt{+} для \texttt{int}, \texttt{+.} для \texttt{float}
    \item \texttt{print\_int}, \texttt{print\_float}, \texttt{print\_string}, ...
  \end{itemize}
\end{itemize}
\end{frame}

\lstset{language=caml}
\begin{frame}[fragile]\frametitle{Введение. Модули в ML}
\begin{itemize}
  \item Язык разделён на два слоя: основной и модульный
  \item Модульный язык более мощный, но требует избыточности и излишней явности деклараций
  \item Слои плохо интегрируются между собой: нельзя динамически выбрать модуль
  \item Нельзя: \begin{lstlisting}
  module Table = if size > threshold then HashMap else TreeMap
  \end{lstlisting}
  \item Можно: \begin{lstlisting}
  module Table = (val (if size > threshold then (module HashMap : MAP) else (module TreeMap : MAP))) : MAP)
  \end{lstlisting}
  \item В некоторых случаях такая интеграция невозможна
\end{itemize}
\end{frame}

\begin{frame}\frametitle{Введение. 1ML}
\begin{itemize}
  \item Было показано \footcite{fing}, что модули могут быть выражены и без использования теории зависимых типов
  \item А именно, что модули можно полностью выразить в System F\textsubscript{$\omega$}
  \item В результате этих исследований был создан экспериментальный диалект 1ML \footcite{1ml}, где основной и модульный слои объединены
  \item В нём действительно есть модули первого класса
  \item 
\end{itemize}
\end{frame}

\lstset{language=haskell}
\begin{frame}[fragile]\frametitle{Обзор литературы. Классы типов}
\begin{lstlisting}
class Show a where
  show :: a -> String

instance Show Int where
  show = showSignedInt

show_twice x = show x ++ show x

show_twice : Show a => a -> string
\end{lstlisting}
\begin{itemize}
  \item Впервые в Haskell\footcite{adhoc}, затем в Agda, Rust, ...
  \item Требуется каноничность~--- не более одного экземпляра для каждого типа
\end{itemize}
\end{frame}

\lstset{language=scala}
\begin{frame}[fragile]\frametitle{Обзор литературы. Неявные параметры}
\begin{lstlisting}
trait Showable [T] { def show (x: T): String }

implicit object IntShowable extends Showable [Int] {
  def show (x: Int) = x.toString
}

def show[T](x : T)(implicit s: Showable [T]): String = {
  s.show(x)
}

show(7)(IntShowable)
show(7)
\end{lstlisting}
Впервые в Scala\footcite{implicits}
\end{frame}

\begin{frame}\frametitle{Обзор литературы. Модульные классы типов}
\begin{itemize}
  \item Попытка применить классы типов в ML\footcite{ml_typeclasses}
  \item Каноничность невозможна в модульном языке
  \item Поэтому решение вводит ряд серьёзных ограничений
  \begin{itemize}
    \item Неявные модули могут быть объявлены только на верхнем уровне
    \item Все модули на верхнем уровне должны быть явно типизированы
    \item На верхнем уровне могут находиться только модули
    \item Все неявные модули должны определять тип \texttt{t}, по которому будет проходить унификация 
  \end{itemize}
\end{itemize}
\end{frame}

\begin{frame}\frametitle{Обзор литературы. Неявные модули}
\begin{itemize}
  \item Попытка применить неявные параметры в OCaml\footcite{white}, вместо обычных параметров используются модули
  \item Не нужна каноничность, нет ограничений как в модульных классах типов
  \item Нет неявных преобразований
  \item Нет приоритета неявных модулей, несколько подходящих кандидатов приводят к ошибке
  \item Есть прототип, но в основной язык интегрировать не получилось
  \item Требуется унификация на модульном уровне, в прототипе недостаточно сильная
\end{itemize}
\end{frame}

\lstset{language=caml}
\begin{frame}[fragile]\frametitle{Обзор литературы. Неявные модули}
\begin{lstlisting}
module type Show = sig
  type t
  val show : t -> string
end

implicit module Show_int = struct
  type t = int
  let show x = string_of_int x
end

implicit module Show_list {S : Show} = struct
  type t = S.t list
  let show x = string_of_list S.show x
end

let show {S : Show} x = S.show x

show 5 (* show {Show_int} 5 *)
show [1;2;3] (* show {Show_list (Show_int)} [1;2;3] *)
\end{lstlisting}
\end{frame}

\begin{frame}\frametitle{Мотивация}
\begin{itemize}
  \item Расширение OCaml неявными модулями~--- сложная задача, требующая огромной и практической, и теоретической работы
  \item Попробуем реализовать неявные модули на более простом диалекте OCaml и попытаемся улучшить слабые стороны предыдущего решения
  \item Выбран 1ML из-за уникального подхода к модульной системе, который может помочь в получении новых результатов
\end{itemize}
\end{frame}

\begin{frame}\frametitle{Цель и задачи}
Цель:
\begin{itemize}
  \item Дополнить язык 1ML поддержкой неявных модулей
\end{itemize}
Задачи:
\begin{itemize}
  \item Реализация неявных модулей, повторяющих функциональность решения для OCaml
  \item Расширение решения новой функциональностью
  \item Тестирование и сравнение с неявными модулями для OCaml
\end{itemize}
\end{frame}

\begin{frame}\frametitle{Общая схема}
\begin{itemize}
  \item Обработка неявных модулей тесно связана с выводом типов, одно нельзя сделать до второго. Поэтому их нужно делать в одной фазе компиляции
  \item Обрабатывая неявную аппликацию, подставим неявную переменную
  \item Зная, какой нужен тип, нужно найти модуль с таким типом
  \item Отложим определение этого модуля, будем обрабатывать несколько неявных переменных за раз
\end{itemize}
\end{frame}

\begin{frame}\frametitle{Поиск модулей}
\begin{itemize}
  \item Бесконечное число модулей
  \item Представим текущее состояние поиска как набор ограничений на тип
  \item Переберём все доступные модули, если подходит под ограничения~--- подставим
  \item Если это функтор, то запустимся рекурсивно с новыми ограничениями
  \item Результат~--- "успех", "нет кандидатов", "неоднозначность", "не завершается"
\end{itemize}
\end{frame}

\lstset{language=caml}
\begin{frame}[fragile]\frametitle{Проверка завершаемости}
\begin{lstlisting}
implicit module Show_it {S : Show} = struct
  type t = S.t
  let show = S.show
end
\end{lstlisting}
\begin{itemize}
  \item Поиск может не завершаться: \texttt{Show\_it(Show\_it(Show\_it...))}
  \item Нужно определять ситуации, когда алгоритм может не завершиться
  \item Будем определять незавершаемость, если между двумя применениями одного и того же модуля входные данные не стали меньше
  \item То есть хотя бы одно ограничение на тип стало меньше, остальные стали не больше
  \item Нужно подождать, не уточнится ли вход из других веток, и если нет, то завершиться
\end{itemize}
\end{frame}

\begin{frame}\frametitle{Генерализация типов}
\begin{itemize}
  \item До какого момента откладывать разрешение неявных переменных?
  \item Чем дольше, тем больше информации
  \item В OCaml ~--- до ближайшей генерализации, то есть до ближайшего \texttt{let}-связывания
  \item Можно пропускать типовые переменные, связанные с неявными переменными, при генерализации
  \item Тогда в теории можно откладывать разрешение до самого конца обработки программы
  \item В реализации откладываем до ближайшего объявления на верхнем уровне
\end{itemize}
\end{frame}

\lstset{language=caml}
\begin{frame}[fragile]\frametitle{Локальные неявные модули}
\begin{lstlisting}
let f = show 5 ^ " " ^
  (let implicit module Show_float = struct
      type t = float
      let show x = string_of_float x
   end in show 3.14)
\end{lstlisting}
\begin{itemize}
\item Так как разрешение неявных переменных отложено, к моменту разрешения некоторые модули могут выйти из контекста
\item Храним дерево из неявных модулей и побочной информации
\item Каждая неявная переменная сопоставляется с вершиной в дереве, может использовать модули на пути от этой вершины до корня
\end{itemize}
\end{frame}

\begin{frame}\frametitle{Порядок разрешения. Мотивация}
\begin{itemize}
  \item В решении для OCaml неявные переменные разрешаются в некотором фиксированном порядке
  \item Это уменьшает полноту решения, например, нельзя реализовать сложение \texttt{int} с \texttt{float}
  \item Будем запускать разрешение неявных переменных несколько раз, с появлением новой информации
  \item Запоминаем и используем предыдущие результаты
\end{itemize}
\end{frame}

\lstset{language=caml}
\begin{frame}[fragile]\frametitle{Порядок разрешения. Пример}
\begin{lstlisting}
module type Num = sig
  type t and u and res
  val ( + ) : t -> u -> res
end;;

let ( + ) {N : Num} = N.( + );;

implicit module Float_Float = struct
  type t = float and u = float and res = float
  let ( + ) = ( +. )
end;;
implicit module Int_Float = struct
  type t = int and u = float and res = float
  let (+) l r = (float_of_int l) +. r
end;;

(* Int_Int и Float_Int §пропущены для краткости§ *)

print_float (1 + 1.1 + 2.5);; (* неоднозначность! *)
\end{lstlisting}
\end{frame}

\begin{frame}\frametitle{Порядок разрешения. Постановка задачи}
\begin{itemize}
  \item Каждая неявная переменная характеризуется своим типом, зависящим от нуля или нескольких типовых переменных: $T_i(x_{a_{i,1}}, \dots, x_{a_{i,n_i}})$
  \item Типовые переменные $x_k$ могут повторяться для разных $T_i$
  \item После поиска модуля по $T_i$ могут быть определены все или некоторые из $x_{a_{i,1}}, \dots, x_{a_{i,n_i}}$
  \item Если поиск модуля вернул "нет кандидатов", то нужно завершить алгоритм. В случаях же "неоднозначность" или "не завершается" нужно подождать новой информации
  \item Для $N$ неявных переменных в худшем случае потребуется $\mathcal{O}(N^2)$ запусков поиска 
\end{itemize}
\end{frame}

\begin{frame}\frametitle{Порядок разрешения. Алгоритм}
\begin{enumerate}
  \item На каждом шаге, если нашлось такое $T_i$, что все $x_{a_{i,j}}$, от которых оно зависит, уникальны~--- обработать его
  \item Если таких не нашлось, обработать любой $T_i$, который ни разу не был обработан
  \item Если и таких не нашлось, обработать $T_i$, с момента последней неудачной обработки которого хотя бы одна из $x_{a_{i,j}}$ была определена
\end{enumerate}
Сложность: $\mathcal{O}(N + K)$ запусков поиска, где $K = \sum\limits_i n_i$
Эвристики:
\begin{itemize}
  \item В (2) обработать сначала $T_i$ с меньшим числом переменных
  \item В (3) обработать сначала $T_i$, про который стало известно больше новой информации
\end{itemize}
\end{frame}

\lstset{language=caml}
\begin{frame}[fragile]\frametitle{Неявные функторы}
Допустим, есть \texttt{Show\_list1} и \texttt{Show\_list2} с одинаковой сигнатурой, нужно выбрать из них явно
\begin{lstlisting}
show {Show_list1 (Show_pair (Show_int Show_bool)}
  [(1, true); (2, false)]

(* §Слишком длинно. В OCaml можно только так§ *)

show {Show_list1} [(1, true); (2, false)]

(* §В 1ML можно поддержать такое§ *)
\end{lstlisting}
Это достигнуто за счёт того, что в 1ML различие между функциями и функторами существенно меньше
\end{frame}

\begin{frame}\frametitle{Тестирование и сравнение}
\begin{itemize}
  \item Нет существующей кодовой базы ни на OCaml, ни на 1ML
  \item Нужно собрать собственный набор тестов
  \item Источники:
  \begin{itemize}
    \item Набор тестов для решения на OCaml и примеры из статей
    \item Тесты на порядок разрешения и на неявные функторы
    \item Тесты, аналогичные коду на других языках с неявными параметрами (Scala)
    \item Реализация стандартных функциональных структур (например, монады)
    \item Тесты, написанные в логической парадигме
  \end{itemize}
\end{itemize}
\end{frame}

\begin{frame}\frametitle{Результаты}
\begin{itemize}
    \item Реализованы неявные модули как расширение компилятора языка 1ML
    \item Решение работает на тестах, на которых работает решение для OCaml
    \item Также работает на тестах, которые в OCaml не поддерживаются
    \begin{itemize}
        \item Порядок разрешения
        \item Неявные функторы
    \end{itemize}
    \item Тестовый набор будет дополняться 
\end{itemize}
\end{frame}

\appendix

\begin{frame}[allowframebreaks]\frametitle{Ссылки}
\printbibliography
\end{frame}

\end{document}
