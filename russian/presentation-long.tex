%% Простая презентация с примером включения программного кода и
%% пошаговых спецэффектов
\documentclass{beamer}
\usetheme{SPbAU}
%\useoutertheme{infolines}
\usepackage{fontspec}
\usepackage{xunicode}
\usepackage{xltxtra}
\usepackage{xecyr}
\usepackage{hyperref}
\setmainfont[Mapping=tex-text]{DejaVu Serif}
\setsansfont[Mapping=tex-text]{DejaVu Sans}
\setmonofont[Mapping=tex-text]{DejaVu Sans Mono}
\usepackage{polyglossia}
\usepackage{appendixnumberbeamer}
\setdefaultlanguage{russian}
\usepackage{graphicx}
\usepackage{listings}
\usepackage{xpatch}
\newcommand{\commentfont}{\ttfamily}
\lstdefinestyle{mycode}{
  belowcaptionskip=1\baselineskip,
  breaklines=true,
  xleftmargin=\parindent,
  showstringspaces=false,
  basicstyle=\footnotesize\ttfamily,
  keywordstyle=\bfseries,
  commentstyle=\itshape\color{gray!40!black},
  stringstyle=\color{red},
  numbers=left,
  numbersep=5pt,
  numberstyle=\tiny\color{gray},
  escapeinside=§§,
  escapebegin=\begin{russian}\commentfont,
  escapeend=\end{russian},
}
\lstset{escapechar=@,style=mycode}


\usepackage[backend=biber, style=authortitle, doi=false, url=false]{biblatex}
\xapptobibmacro{cite}{\setunit{\nametitledelim}\printfield{year}}{}{}
\addbibresource{presentation-long.bib}

\begin{document}
\title[Неявные модули в 1ML]{Переоснащение неявных модулей для языка 1ML }
\author[Трилис А.А.]{Трилис Алексей Андреевич\\{\footnotesize\textcolor{gray}{научный руководитель: к.ф.-м.н. Д.А. Березун}}}
\institute{НИУ ВШЭ --- Санкт-Петербург}
\frame{\titlepage}

\begin{frame}\frametitle{Введение. Ad hoc полиморфизм}
\begin{itemize}
  \item Ad hoc полиморфизм~--- свойство языка, позволяющее функциям иметь различную семантику в зависимости от типов аргументов
  \item Например, \texttt{print} и \texttt{+}
  \item В процедурных и объектно-ориентированных языках обычно достигается перегрузкой
  \item Но в языках с мощным выводом типов нужны более сложные методы
\end{itemize}
\end{frame}

\begin{frame}\frametitle{Введение. Семейство языков ML}
\begin{itemize}
  \item ML, OCaml, SML, F\#, ...
  \item Функциональные языки с мощным выводом типов
  \item Продвинутая система модулей, основанная на теории зависимых типов \footcite{dependent_types}
  \item Активно используется в разработке и исследовании языков программирования
  \item А также в верификации, финансах, веб-разработке и других областях
  \item Отсутствует ad hoc полиформизм:
  \begin{itemize}
    \item \texttt{+} для \texttt{int}, \texttt{+.} для \texttt{float}
    \item \texttt{print\_int}, \texttt{print\_float}, \texttt{print\_string}, ...
  \end{itemize}
\end{itemize}
\end{frame}

\lstset{language=caml}
\begin{frame}[fragile]\frametitle{Введение. Модули в ML}
\begin{itemize}
  \item Язык разделён на два слоя: основной и модульный
  \item Модульный язык более мощный, но требует избыточности и излишней явности деклараций
  \item Слои плохо интегрируются между собой: нельзя динамически выбрать модуль
  \item Нельзя: \begin{lstlisting}
  module Table = if size > threshold then HashMap else TreeMap
  \end{lstlisting}
  \item Можно: \begin{lstlisting}
  module Table = (val (if size > threshold then (module HashMap : MAP) else (module TreeMap : MAP))) : MAP)
  \end{lstlisting}
  \item В некоторых случаях такая интеграция невозможна
\end{itemize}
\end{frame}

\begin{frame}\frametitle{Введение. 1ML}
\begin{itemize}
  \item Было показано \footcite{fing}, что модули могут быть выражены и без использования теории зависимых типов
  \item А именно, что модули можно полностью выразить в System F\textsubscript{$\omega$}
  \item В результате этих исследований был создан экспериментальный диалект 1ML \footcite{1ml}
\end{itemize}
\end{frame}

\lstset{language=haskell}
\begin{frame}[fragile]\frametitle{Обзор литературы. Классы типов}
\begin{lstlisting}
class Show a where
  show :: a -> String

instance Show Int where
  show = showSignedInt

show_twice x = show x ++ show x

show_twice : Show a => a -> string
\end{lstlisting}
\begin{itemize}
  \item Впервые в Haskell\footcite{adhoc}, затем в Agda, Rust, ...
  \item Требуется каноничность~--- не более одного экземпляра для каждого типа
\end{itemize}
\end{frame}

\lstset{language=scala}
\begin{frame}[fragile]\frametitle{Обзор литературы. Неявные параметры}
\begin{lstlisting}
trait Showable [T] { def show (x: T): String }

implicit object IntShowable extends Showable [Int] {
  def show (x: Int) = x.toString
}

def show[T](x : T)(implicit s: Showable [T]): String = {
  s.show(x)
}

show(7)(IntShowable)
show(7)
\end{lstlisting}
Впервые в Scala\footcite{implicits}
\end{frame}

\begin{frame}\frametitle{Обзор литературы. Модульные классы типов}
\begin{itemize}
  \item Попытка применить классы типов в ML\footcite{ml_typeclasses}
  \item Каноничность невозможна в модульном языке
  \item Поэтому решение вводит ряд серьёзных ограничений
  \begin{itemize}
    \item
  \end{itemize}
\end{itemize}
\end{frame}

\begin{frame}\frametitle{Обзор литературы. Неявные модули}
\begin{itemize}
  \item 
\end{itemize}
\end{frame}

\lstset{language=caml}
\begin{frame}[fragile]\frametitle{Обзор литературы. Неявные модули}
\begin{lstlisting}
module type Show = sig
  type t
  val show : t -> string
end

implicit module Show_int = struct
  type t = int
  let show x = string_of_int x
end

implicit module Show_list {S : Show} = struct
  type t = S.t list
  let show x = string_of_list S.show x
end

let show {S : Show} x = S.show x

show 5 (* show {Show_int} 5 *)
show [1;2;3] (* show {Show_list (Show_int)} [1;2;3] *)
\end{lstlisting}
\end{frame}

\begin{frame}\frametitle{Мотивация}
\begin{itemize}
  \item 
\end{itemize}
\end{frame}

\begin{frame}\frametitle{Цель и задачи}
Цель:
\begin{itemize}
  \item Дополнить язык 1ML поддержкой неявных модулей
\end{itemize}
Задачи:
\begin{itemize}
  \item Реализация неявных модулей, повторяющих функциональность решения для OCaml
  \item Расширение решения новой функциональностью
  \item Тестирование и сравнение с неявными модулями для OCaml
\end{itemize}
\end{frame}

\begin{frame}\frametitle{Общая схема}
  
\end{frame}

\begin{frame}\frametitle{Поиск модулей}
\begin{itemize}
  \item Результат~--- "успех", "нет кандидатов", "неоднозначность", "не завершается"
\end{itemize}
\end{frame}

\begin{frame}\frametitle{Генерализация типов}

\end{frame}

\lstset{language=caml}
\begin{frame}[fragile]\frametitle{Проверка завершаемости}
\begin{lstlisting}
implicit module Show_it {S : Show} = struct
  type t = S.t
  let show = S.show
end
\end{lstlisting}

\end{frame}

\lstset{language=caml}
\begin{frame}[fragile]\frametitle{Локальные неявные модули}
\begin{lstlisting}
let f = show 5 ^ " " ^
  (let implicit module Show_float = struct
      type t = float
      let show x = string_of_float x
   end in show 3.14)
\end{lstlisting}
\begin{itemize}
\item Так как разрешение неявных переменных отложено, к моменту разрешения некоторые модули могут выйти из контекста
\item Храним дерево из неявных модулей и побочной информации
\item Каждая неявная переменная сопоставляется с вершиной в дереве, может использовать модули на пути от этой вершины до корня
\end{itemize}
\end{frame}


\begin{frame}\frametitle{Порядок разрешения. Мотивация}
\begin{itemize}
  \item
\end{itemize}
\end{frame}

\lstset{language=caml}
\begin{frame}[fragile]\frametitle{Порядок разрешения. Пример}
\begin{lstlisting}
module type Num = sig
  type t and u and res
  val ( + ) : t -> u -> res
end;;

let ( + ) {N : Num} = N.( + );;

implicit module Float_Float = struct
  type t = float and u = float and res = float
  let ( + ) = ( +. )
end;;
implicit module Int_Float = struct
  type t = int and u = float and res = float
  let (+) l r = (float_of_int l) +. r
end;;

(* Int_Int и Float_Int §пропущены для краткости§ *)

print_float (1 + 1.1 + 2.5);; (* неоднозначность! *)
\end{lstlisting}
\end{frame}



\begin{frame}\frametitle{Порядок разрешения. Постановка задачи}
\begin{itemize}
  \item Каждая неявная переменная характеризуется своим типом, зависящим от нуля или нескольких типовых переменных: $T_i(x_{a_{i,1}}, \dots, x_{a_{i,n_i}})$
  \item Типовые переменные $x_k$ могут повторяться для разных $T_i$
  \item После поиска модуля по $T_i$ могут быть определены все или некоторые из $x_{a_{i,1}}, \dots, x_{a_{i,n_i}}$
  \item Если поиск модуля вернул "нет кандидатов", то нужно завершить алгоритм. В случаях же "неоднозначность" или "не завершается" нужно подождать новой информации
  \item Для $N$ неявных переменных в худшем случае потребуется $\mathcal{O}(N^2)$ запусков поиска 
\end{itemize}
\end{frame}

\begin{frame}\frametitle{Порядок разрешения. Алгоритм}
\begin{enumerate}
  \item На каждом шаге, если нашлось такое $T_i$, что все $x_{a_{i,j}}$, от которых оно зависит, уникальны~--- обработать его
  \item Если таких не нашлось, обработать любой $T_i$, который ни разу не был обработан
  \item Если и таких не нашлось, обработать $T_i$, с момента последней неудачной обработки которого хотя бы одна из $x_{a_{i,j}}$ была определена
\end{enumerate}
Сложность: $\mathcal{O}(N + K)$ запусков поиска, где $K = \sum\limits_i n_i$
Эвристики:
\begin{itemize}
  \item В (2) обработать сначала $T_i$ с меньшим числом переменных
  \item В (3) обработать сначала $T_i$, про который стало известно больше новой информации
\end{itemize}
\end{frame}

\lstset{language=caml}
\begin{frame}[fragile]\frametitle{Неявные функторы}
Допустим, есть \texttt{Show\_list1} и \texttt{Show\_list2} с одинаковой сигнатурой, нужно выбрать из них явно
\begin{lstlisting}
show {Show_list1 (Show_pair (Show_int Show_bool)}
  [(1, true); (2, false)]

(* §Слишком длинно. В OCaml можно только так§ *)

show {Show_list1} [(1, true); (2, false)]

(* §В 1ML можно поддержать такое§ *)
\end{lstlisting}
Это достигнуто за счёт того, что в 1ML различие между функциями и функторами существенно меньше
\end{frame}

\begin{frame}\frametitle{Тестирование и сравнение}
\begin{itemize}
  \item Нет существующей кодовой базы ни на OCaml, ни на 1ML
  \item Нужно собрать собственный набор тестов
  \item Источники:
  \begin{itemize}
    \item Набор тестов для решения на OCaml и примеры из статей
    \item Тесты на порядок разрешения и на неявные функторы
    \item Тесты, аналогичные коду на других языках с неявными параметрами (Scala)
    \item Реализация стандартных функциональных структур (например, монады)
    \item Тесты, написанные в логической парадигме
  \end{itemize}
\end{itemize}
\end{frame}

\begin{frame}\frametitle{Результаты}
\begin{itemize}
    \item
\end{itemize}
\end{frame}

\appendix

\begin{frame}[allowframebreaks]\frametitle{Ссылки}
\printbibliography
\end{frame}

\end{document}
